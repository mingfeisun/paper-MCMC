\section{Connections to reinforcement learning}
We now show optimizing Equation~\ref{equ:general_objective} is related to actor-critic algorithms.

\subsection{Actor-Critic optimization}
Assume we parametrize $\pi$ and $f$ with $\theta$ and $\omega$ respectively. 
In order to optimize Equation~\ref{equ:general_objective}, we can iteratively solve the maximization and the minimization. 
We now show that how this $\min$-$\max$ optimization problem resembles the actor-critic algorithms in reinforcement learning. 

\paragraph{Maximization} First, the $\sup_{f\in\mathcal{F}}$ can be directly optimized by taking the gradient of $f_{\omega}(s, a)$ and $\tilde{f}_{\omega}(s, a)$. 
\begin{equation*}
\nabla_{\omega} \Big[ \mathbb{E}_{(s, a)\sim d_{\pi_{\theta}}} \big[ f_{\omega}(s, a) \big] + \mathbb{E}_{(s, a)\sim d_*} \big[ \tilde{f}_{\omega}(s, a) \big] \Big]
= \mathbb{E}_{(s, a)\sim d_{\pi_{\theta}}} \big[ \nabla_{\omega} f_{\omega}(s, a) \big] + \mathbb{E}_{(s, a)\sim d_*} \big[ \nabla_{\omega} \tilde{f}_{\omega}(s, a) \big].
\end{equation*}

\paragraph{Minimization} Second, the $\min_{\pi}$ is with respect to $\pi$, which is parameterized with $\theta$ and only included in the first term.
\begin{equation*}
\nabla_{\theta} \Big[ \mathbb{E}_{(s, a)\sim d_{\pi_{\theta}}} \big[ f_{\omega}(s, a) \big] + \mathbb{E}_{(s, a)\sim d_*} \big[ \tilde{f}_{\omega}(s, a) \big] \Big]
= \nabla_{\theta} \mathbb{E}_{(s, a)\sim d_{\pi_{\theta}}} \big[ f_{\omega}(s, a) \big] 
\end{equation*}
Since $\pi_{\theta}$ is within samples, directly computing the gradient would be intractable. However, recall that in reinforcement learning the objective, $J(\theta)=\mathbb{E}_{s\sim d_{\pi_{\theta}}(s), a\sim\pi_{\theta}}[r(s, a)]$, shares the same form. If we interpret $f_{\omega}(s, a)$ as a reward function, we can then compute the gradient. 
\begin{theorem}\label{theo:policy_gradient}
For any state occupancy $d_{\pi_{\theta}}$, in either the average or discounted cases, and any function $f_{\omega}$, 
\begin{equation*}
\nabla_\theta \mathbb{E}_{(s, a)\sim d_{\pi_{\theta}}}\big[ f_{\omega}(s, a) \big] = \mathbb{E}_{(s, a)\sim d_{\pi_{\theta}}} \big[ \Psi_{\omega}(s, a) \cdot \nabla_\theta \log \pi_{\theta}(s, a) \big], 
\end{equation*}
where $\Psi_{\omega}(s, a)$ is a function defined according to $f_{\omega}(s, a)$. 
\end{theorem}

\begin{proof}
Note that $\phi(s, a)$ is also a function independent of $\pi$. 
This proof is similar to the proof of policy gradient theorem in terms that we view $\phi(s, a)$ as the reward function, i.e., $\hat{r}(s, a) := \phi(s, a)$. 

In the average reward formulation, $d_\pi$ is defined as $d_{\pi}=\lim_{t\leftarrow\infty}Pr(s_t=s, a_t=a|s_0, \pi)$, which we assume exists and is independent of $s_0$ for all policies. 
The $\Psi(s, a)$ is defined as 
\begin{equation*}
\Psi(s, a) = \sum_{t=1}^{\infty} \mathbb{E}[\phi(s_t, a_t) - \rho(\pi) | s_0 = s, a_0 =a, \pi], \quad \forall s\in\mathcal{S}, a\in\mathcal{A}, 
\end{equation*}
where $\rho(\pi) = \mathbb{E}_{(s, a)\sim d_{\pi_{\theta}}}\big[ \phi(s, a) \big] $. 

In the discounted reward formulation, $d_\pi$ is defined as $d_{\pi}=\sum_{t=0}^{\infty}\gamma^t Pr(s_t=s, a_t=a|s_0, \pi)$
The $\Psi(s, a)$ is defined as 
\begin{equation*}
\Psi(s, a) = \mathbb{E}\big[ \sum_{k=1}^{\infty} \gamma^{k-1} \phi(s_{t+k}, a_{t+k}) | s_t = s, a_t =a, \pi \big], \quad \forall s\in\mathcal{S}, a\in\mathcal{A}, 
\end{equation*}
where $\gamma\in[0, 1]$ is a discount rate ($\gamma=1$ is allowed only in episodic tasks). 
\end{proof}


\subsection{Deterministic adversarial imitation}
Assume policy $\pi_{\theta}$ is deterministic

\begin{align*}
& \nabla_{\theta} \Big[ \mathbb{E}_{(s, a)\sim d_{\pi_{\theta}}} \big[ \phi_{\omega}(s, a) \big] + \mathbb{E}_{(s, a)\sim d_*} \big[ f_{\phi_{\omega}}(s, a) \big] \Big] \\
= & \nabla_{\theta} \mathbb{E}_{(s, a)\sim d_{\pi_{\theta}}} \big[ \phi_{\omega}(s, a) \big] \\ 
= & \mathbb{E}_{s\sim d_{\pi_{\theta}}(s)} \big[  \nabla_\theta \pi_{\theta}(s, a) \cdot \nabla_a \Psi(s, a)\big|_{a=\pi_{\theta}(s)}\big], 
\end{align*}

The algorithm runs as follows:

\begin{algorithm}
\caption{Deterministic Adversarial Imitation Learning}
\begin{algorithmic}[1] %enables line numbers
\STATE Initialize critic network $Q_{\omega_1}$, $Q_{\omega_2}$, actor network $\pi_{\theta}$ and reward network $r_{\phi}$ wth random parameters $\omega_1$, $\omega_2$, $\theta$ and $\phi$
\STATE Initialize target networks $\omega_1^\prime\rightarrow\omega_1$, $\omega_2^\prime\rightarrow\omega_2$, $\theta^\prime\rightarrow\theta$,
\STATE Initialize replay buffer $\mathcal{B}$
\STATE Initialize demonstration set $\mathcal{D}$
\FOR{$t=1$ to $T$}
\STATE Select action with exploration noise $a\sim\pi_{\theta} + \epsilon$, $\epsilon\sim \mathcal{N}(0, \sigma)$ and observe new state $s^\prime$
\STATE Store transition tuple $(s, a, s^\prime)$ in $\mathcal{B}$
\STATE Sample mini-batch of $N$ transitions $(s, a, s^\prime)$ from $\mathcal{B}$
\IF{$t$ mod $d_r$}
\STATE Sample mini-batch of $N$ transitions $(s, a, s^\prime)$ from $\mathcal{D}$
\STATE Optimize $r_{\phi}$ based on the objective
\ENDIF
\STATE $\tilde{a} \leftarrow \pi_{\theta^\prime}+\epsilon$, $\epsilon\sim \text{clip}(\mathcal{N}(0, \tilde{\sigma}, -c, c))$
\STATE $y\rightarrow r_{\phi}+\gamma \min_{i=1, 2}Q_{\phi_i}^\prime(s^\prime, \tilde{a})$
\STATE Update critics $\omega_i\leftarrow \arg\min_{\omega_i} N^{-1}\sum(y - Q_{\omega_i}(s, a))^2$
\IF{$t$ mod $d_\pi$}
\STATE Update $\theta$ by the deterministic policy gradient
\STATE Update target networks: $\omega_i^\prime\leftarrow \tau \omega_i + (1-\tau)\omega_i^\prime$, $\theta^\prime \leftarrow \tau \theta + (1-\tau)\theta $
\ENDIF
\ENDFOR
\end{algorithmic}
\end{algorithm}
